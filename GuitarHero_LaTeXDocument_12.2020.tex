\documentclass[12pt, letterpaper]{article}
\usepackage[utf8]{inputenc}
\usepackage{graphicx}
\usepackage{amsmath}

\title{Creating a Test Statistic to Find if the Number of Misses in a Song from Guitar Hero is Random}
\author{Shannon Coyle, Samantha Colucci, and Brianna Cirillo}
\date{December 2020}

\begin{document}
\maketitle

\section{INTRODUCTION}
The game, Guitar Hero, collects information regarding the number of “hits” recorded by the player.  This poses the question about the randomness of the misses in a song and if they are correlated to the difficulty of the part of the song.

The study investigates the randomness of different songs using varying methods.  These methods looked at the number of miss streaks over the total number of misses, distances between each miss, and using the runs test.  The resampling of the data sets was performed through parametric bootstrapping, permutation tests, and regular bootstrapping. Using resampling and our methods together allowed us to assess the randomness of a given song based on the results. 

This report explains the methodology used in order to test the randomness of a song. The methods and resampling used throughout the project gave results for the empirical type 1 error and the power of the test. These methods were applied to different songs, therefore, returning different results. This report will also look at the limitations and future ideas for the project, and how some of our methods worked well while others did not.

\section{METHODOLOGY}
\subsection{Method 1}

\subsection{Method 2}

\subsection{Runs Test}


\section{EMPIRICAL TYPE 1 ERROR AND POWER SIMULATIONS}

\subsection{Parametric Bootstrap}
\subsection{Permutation Test}
\subsection{Regular Bootstrap}


\section{APPLICATION}


\section{DISCUSSION/LIMITATIONS/IDEAS FOR FURTHER RESEARCH}


\section{APPENDIX}


\end{document}